%%%%%%%%%%%%%%%%%%%%%%%%%%%%%%%%%%%%%%%%%%%%%%%%%%%%
% Document type, global settings, and packages
%%%%%%%%%%%%%%%%%%%%%%%%%%%%%%%%%%%%%%%%%%%%%%%%%%%%

\documentclass[11pt, twoside]{report}
\usepackage[a4paper,hmargin=3.5cm,vmargin=5.3cm,head=1.5cm,foot=3.2cm]{geometry}
\usepackage[bookmarks=true, hidelinks]{hyperref}     % hyperlink
\usepackage{url}          % simple URL typesetting

% https://tex.stackexchange.com/questions/47576/combining-ifxetex-and-ifluatex-with-the-logical-or-operation
\usepackage{ifxetex,ifluatex}
\newif\ifxetexorluatex
\ifxetex
  \xetexorluatextrue
\else
  \ifluatex
    \xetexorluatextrue
  \else
    \xetexorluatexfalse
  \fi
\fi

\ifxetexorluatex
  % \usepackage{stix}
  \usepackage{fontspec}
  \usepackage{unicode-math}

  \defaultfontfeatures{Scale=MatchLowercase,Ligatures=TeX}
  % \setmainfont{STIX Two Text}
  \setmainfont{Noto Serif}
  \setmathfont{TeX Gyre Schola Math}
  \usepackage[scale=0.8]{noto-mono}
  \setmonofont{Noto Sans Mono}
  % small url fonts
  \renewcommand{\UrlFont}{\ttfamily\small}

  \ifxetex
    % https://gist.github.com/dlimpid/5454229
    % Use xeCJK in XeLaTeX
    \usepackage{xeCJK}
    \xeCJKsetup{%
      CJKspace=true,% % true이면 띄어쓰기 사용. 중국어, 일어는 필요 없을수도
      % CJKmath=true,%  % true면 math environment 안에서 CJK 글자 사용
      CJKecglue={}%   % Western과 CJK 사이의 공백 지정: {}로 간격을 없앰
    }
    \setCJKmainfont[
      Path=fonts/,
      Extension=.otf,
      BoldFont=NotoSerifCJKkr-Bold,
      AutoFakeSlant,
      Ligatures=TeX]{NotoSerifCJKkr-Regular.otf}
    \setCJKsansfont[
      Path=fonts/,
      Extension=.otf,
      BoldFont=NotoSerifCJKkr-Bold,
      AutoFakeSlant,
      Ligatures=TeX]{NotoSerifCJKkr-Regular.otf}
    \setCJKmonofont[
      Path=fonts/,
      Extension=.otf,
      BoldFont=NotoSerifCJKkr-Bold,
      AutoFakeSlant,
      Ligatures=TeX]{NotoSerifCJKkr-Regular.otf}
  \fi

  \ifluatex
    \usepackage[hangul]{luatexko}
    \setmainhangulfont[
      Path=fonts/,
      UprightFont=*-Regular,
      BoldFont=*-Bold,
      Extension=.otf,
      AutoFakeSlant,
      Ligatures=TeX]{NotoSerifCJKkr}
    \hangulbyhangulfont=1
    \setmainfont{Noto Serif}
    \setmathfont{TeX Gyre Schola Math}
    \usepackage[scale=0.8]{noto-mono}
    \setmonofont{Noto Sans Mono}
  \fi
\else
  % pdflatex doesn't work due to siunitx and kotex conflict
  % https://t.co/PgBfFh9rTU?amp=1
  \usepackage{amssymb,amsmath,amsthm,amsfonts}
  \usepackage[T1]{fontenc}
  \usepackage[utf8]{inputenc}

  % \usepackage{CJKutf8}
  \usepackage{textcomp} % provide euro and other symbols

  % use nanum font in Korean
  \usepackage{dhucs-nanumfont}
  % otherwise, use Noto fonts
  \usepackage{noto}
\fi

\usepackage[finemath]{kotex} % Use kotex in PDFLaTeX

\ifxetex
  \XeTeXinputnormalization=1
\fi

\usepackage{listings}
\usepackage{color}

\definecolor{dkgreen}{rgb}{0,0.6,0}
\definecolor{gray}{rgb}{0.5,0.5,0.5}
\definecolor{mauve}{rgb}{0.58,0,0.82}

\lstset{frame=tb,
  language=Java,
  aboveskip=6mm,
  belowskip=6mm,
  showstringspaces=false,
  columns=flexible,
  basicstyle={\small\ttfamily},
  numbers=none,
  numberstyle=\tiny\color{gray},
  keywordstyle=\color{blue},
  commentstyle=\color{dkgreen},
  stringstyle=\color{mauve},
  breaklines=true,
  breakatwhitespace=true,
  tabsize=3
}

\usepackage{setspace}   %use this package to set linespacing as desired
\doublespacing
\usepackage{afterpage}    % separate page for floats and tables
\usepackage{titlefoot}    % keywords on footnote
\usepackage{bookmark}
\usepackage[explicit]{titlesec}  %title control and formatting
\usepackage[titles]{tocloft}  %table of contents control and formatting
\usepackage[page]{appendix}  %for appendices
\usepackage[normalem]{ulem}  %for underlined section titles

\usepackage{graphicx}  %for images and plots
\usepackage[figuresright]{rotating}  %for rotated, landscape images
\usepackage{adjustbox}    % fit table to page width

\usepackage{booktabs}     % proessional-quality tables
\usepackage{multicol} 		% use multicol in table
\usepackage{multirow} 		% use multirow in table

\usepackage[super]{nth}		% th
\usepackage{siunitx}		  % SI unit

\usepackage[noabbrev,nameinlink]{cleveref}		% for multiple cross-reference
\usepackage[intoc]{nomencl}  % nomenclature
\makenomenclature
\setlength{\nomitemsep}{-\parsep}
\setlength{\nomlabelwidth}{1.7cm}

%%%%%%%%%%%%%%%%%%%%%%
% Start of Document
%%%%%%%%%%%%%%%%%%%%%%

\begin{document}

%%%%%%%%%%%%%%%%%%%%%%%%%%%%%%%%%%%%%
% Title Page
%%%%%%%%%%%%%%%%%%%%%%%%%%%%%%%%%%%%%

%% Define your thesis title, your name, your school, and your month and year of graduation here
\newgeometry{top=7.13cm,bottom=4.75cm,right=4.17cm,left=4.17cm}
\newcommand{\coverTitle}{Dissertation Title}
\newcommand{\coverSubTitle}{Dissertation SubTitle}
\newcommand{\coverName}{Name}

\newcommand{\thesisTitle}{Dissertation Title}
\newcommand{\mythesisTitle}{Dissertation Title}
\newcommand{\yourName}{Keum Jungbin}
\newcommand{\yourMonth}{March}
\newcommand{\yourYear}{2023}

%%%%%%%%%%%%%%%%%%%%%%%%%%%%%%%%%%%%%%%%%%%%%%%%%%%%%%%%%
% Do not edit these lines unless you wish to customize
% the template
%%%%%%%%%%%%%%%%%%%%%%%%%%%%%%%%%%%%%%%%%%%%%%%%%%%%%%%%%

\begin{titlepage}
\clearpage\thispagestyle{empty}
\begin{center}

{\fontsize{20pt}{20pt} \selectfont \coverTitle \vspace{3.5cm} \par}
% % Enable line below when subtitle exists
% {\fontsize{20pt}{20pt} \selectfont \coverTitle \vspace{0.3cm} \par}
% {\fontsize{16pt}{16pt} \selectfont -\coverSubTitle- \vspace{3.5cm} \par}

{\fontsize{16pt}{20pt} \selectfont \coverName \vspace{3.5cm} \par}
{\fontsize{16pt}{20pt} \selectfont
대건고등학교}
\vfill

\end{center}
\end{titlepage}




%%%%%%%%%%%%%%%%%%%%%%%%%%%%%%%%%%%%%
% Table of Contents
%%%%%%%%%%%%%%%%%%%%%%%%%%%%%%%%%%%%%
\pagenumbering{roman}
\setcounter{page}{1} % set the page number appropriately based on the number of intro pages
% Format for Table of Contents
\renewcommand{\cftchapdotsep}{\cftdotsep}  %add dot separators
\renewcommand{\cftchapfont}{\bfseries}  %set title font weight
\renewcommand{\cftchappagefont}{}  %set page number font weight
\renewcommand{\cftchappresnum}{제 }
\renewcommand{\cftchapaftersnum}{장}
\renewcommand{\cftchapnumwidth}{6em}
\renewcommand{\cftchapafterpnum}{\vskip\baselineskip} %set correct spacing for entries in single space environment
\renewcommand{\cftsecafterpnum}{\vskip\baselineskip}  %set correct spacing for entries in single space environment
\renewcommand{\cftsubsecafterpnum}{\vskip\baselineskip} %set correct spacing for entries in single space environment
\renewcommand{\cftsubsubsecafterpnum}{\vskip\baselineskip} %set correct spacing for entries in single space environment

%format title font size and position (this also applys to list of figures and list of tables)
\titleformat{\chapter}[display]
{\normalfont\bfseries\filcenter}{\chaptertitlename\ \thechapter}{0pt}{\MakeUppercase{#1}}

\renewcommand\contentsname{Table of Contents}

\begin{singlespace}
\tableofcontents
\end{singlespace}

\currentpdfbookmark{Table of Contents}{TOC}

\clearpage

%%%%%%%%%%%%%%%%%%%%%%%%%%%%%%%%%%%%%
% List of figures and tables
%%%%%%%%%%%%%%%%%%%%%%%%%%%%%%%%%%%%%

\addcontentsline{toc}{chapter}{List of Tables}
\begin{singlespace}
	\setlength\cftbeforetabskip{\baselineskip}  %manually set spacing between entries
	\listoftables
\end{singlespace}

\clearpage

\addcontentsline{toc}{chapter}{List of Figures}
\begin{singlespace}
\setlength\cftbeforefigskip{\baselineskip}  %manually set spacing between entries
\listoffigures
\end{singlespace}

\clearpage

\printnomenclature
\clearpage

\currentpdfbookmark{Title Page}{titlePage}  %add PDF bookmark for this page

%%%%%%%%%%%%%%%%%%%%%%%%%%%%
%
% Chapters
%
%%%%%%%%%%%%%%%%%%%%%%%%%%%%

%%%%%%%%%%%%%%%%%%%%%%
% formatting
%%%%%%%%%%%%%%%%%%%%%%

% resume page numbering for rest of document
\clearpage
\pagenumbering{arabic}
\setcounter{page}{1} % set the page number appropriately

% Adjust chapter title formatting
\newcommand{\chapternamefont}{\scshape\Large}% Chapter name font
\newcommand{\chaptertitlefont}{\LARGE\bfseries}% Chapter title font
\titleformat{\chapter}[display]
{\normalfont\bfseries\chapternamefont\raggedright}{\MakeUppercase\chaptertitlefont\chaptertitlename\ \thechapter}{0pt}{\MakeUppercase{#1}}  %spacing between titles
\titlespacing*{\chapter}
  {0pt}{\topskip}{30pt}	%controls vertical margins on title

% Adjust section title formatting
\titleformat{\section}{\normalfont\bfseries}{\thesection}{1em}{#1}

% Adjust subsection title formatting
% \titleformat{\subsection}{\normalfont}{\uline{\thesubsection}}{0em}{\uline{\hspace{1em}#1}}
\titleformat{\subsection}{\normalfont\bfseries}{\thesubsection}{1em}{#1}

% Adjust subsubsection title formatting
\titleformat{\subsubsection}{\normalfont\itshape}{\thesubsection}{1em}{#1}

%%%%%%%%%%%%%%%%
% Chapter 1
%%%%%%%%%%%%%%%%

\chapter{서론 및 이론적 배경}
\label{chap:introduction}

ReactDOM에서 메인 페이지로 로딩하는 코드

\begin{lstlisting}
  // index.tsx
  import React from "react";
  import ReactDOM from "react-dom";
  import App from "./App";
  
  // 이것은 주석입니다.
  ReactDOM.render(<App />, document.getElementById("app"));
\end{lstlisting}


\end{document}